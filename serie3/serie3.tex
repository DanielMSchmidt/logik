\documentclass[12pt, a4paper]{article}
\usepackage{url,graphicx,tabularx,array,geometry}
\usepackage[utf8]{inputenc}
\usepackage{paralist}
\usepackage{latexsym}
\usepackage{fancyhdr}
\usepackage{ textcomp }
\usepackage{ mathrsfs }
\usepackage{ dsfont }
\pagestyle{fancy}

\usepackage{amsmath}
\usepackage{amsfonts}
\usepackage{amssymb}


\setlength{\parskip}{1ex} %--skip lines between paragraphs
\setlength{\parindent}{0pt} %--don't indent paragraphs

%-- Commands for header
\newcommand{\bs}{\ensuremath{\backslash}}
\renewcommand{\title}[1]{\textbf{#1}\\}
\renewcommand{\line}{\begin{tabularx}{\textwidth}{X>{\raggedleft}X}\hline\\\end{tabularx}\\[-0.5cm]}
\newcommand{\leftright}[2]{\begin{tabularx}{\textwidth}{X>{\raggedleft}X}#1%
& #2\\\end{tabularx}\\[-0.5cm]}
%\linespread{2} %-- Uncomment for Double Space
\begin{document}
\renewcommand{\headrulewidth}{0pt}
\fancyhf{}
\fancyhead[L]{
\leftright{\textbf{Logik}}{Hannes Strubel, Daniel Schmidt}
\line
\leftright{\textbf{Gruppe: Mo, 8:15 - 9:45}}{Serie 3}
}
\fancyfoot[C]{\thepage}

\section*{NOFO.1}

\subsection*{1)}
\subsubsection*{Basiszuweisung}
Wenn $\varphi \in V_{AL}$ gilt, so gilt auch $\varphi \in$ MNF
\subsubsection*{Induktion}
Es gilt $J(\varphi_1,...,\varphi_{n-1}) \in$ MNF, wenn $J \in \{\neg, \wedge, \vee \}$ und gilt $\forall i \in \mathbb{N}_{< n}: \varpi_i \in$ MNF.
\subsection*{2)}
AL2MNF($\varphi$)\\
\noindent\hspace*{5mm} if $\varphi \in \{ \top, \bot , X_i \}$\\  
\noindent\hspace*{10mm} return $\varphi$\\    
\noindent\hspace*{5mm} endif\\
\noindent\hspace*{5mm} let $\varphi = J(\varphi_0,...,\varphi_{n-1})$\\
\noindent\hspace*{5mm} for (i = 0; i $\le$ n-1; i++)\\
\noindent\hspace*{10mm} let $\vartheta_i =$ AC2MNF($\varphi_i$)\\
\noindent\hspace*{5mm} end\\
\noindent\hspace*{5mm} case J of\\
\noindent\hspace*{10mm} $\leftrightarrow$ return $(\neg \vartheta_0 \wedge \neg \vartheta_1) \vee (\vartheta_0 \wedge \vartheta_1)$\\
\noindent\hspace*{10mm} $\rightarrow$ return $\neg( \vartheta_0 \neq \vartheta_1)$\\
\noindent\hspace*{10mm} $\uparrow$ return $\neg (\vartheta_0 \wedge \vartheta_1)$\\
\noindent\hspace*{10mm} $\nleftrightarrow$ return $(\neg \vartheta_0 \wedge \vartheta_1) \vee (\vartheta_0 \wedge \neg \vartheta_1)$\\
\noindent\hspace*{5mm} end\\
end

\section*{NNF.4}
Nein, darauf kann nicht verzichtet werden. Die Bedingung um darauf zu verzichten wäre, dass man Terme mit $\top$ und $\perp$ in Verbindung mit den erlaubten Junktoren umformbar sind, sodass $\top$ und $\perp$ wegfallen. Da diese aber bei der Verundung, beziehungsweise bei der Veroderung nicht wegfallen ist es nicht möglich darauf zu verzichten.
\section*{SATS.2}
\subsection*{Gegenbeispiel}
Sei $\varphi = X_0 \in V_{AL}, \vartheta = \neg X_0 \in V_{AL}$, so gilt:
\begin{equation}
\begin{split}
&erfb(\varphi) = 1\\
&erfb(\vartheta) = 1\\
\end{split}
\end{equation}
also gilt: "$erfb(\varphi)$ und $erfb(\vartheta) = 1$", aber
\begin{equation}
\begin{split}
&erfb(\varphi \wedge \vartheta)\\
&=erfb( X_0 \wedge \neg X_0)\\
&=0
\end{split}
\end{equation}
\end{document}