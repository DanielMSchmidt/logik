\documentclass[12pt, a4paper]{article}
\usepackage{url,graphicx,tabularx,array,geometry}
\usepackage[utf8]{inputenc}
\usepackage{paralist}
\usepackage{latexsym}
\usepackage{fancyhdr}
\usepackage{ textcomp }
\usepackage{ mathrsfs }
\usepackage{ dsfont }
\pagestyle{fancy}

\usepackage{amsmath}
\usepackage{amsfonts}
\usepackage{amssymb}


\setlength{\parskip}{1ex} %--skip lines between paragraphs
\setlength{\parindent}{0pt} %--don't indent paragraphs

%-- Commands for header
\newcommand{\bs}{\ensuremath{\backslash}}
\renewcommand{\title}[1]{\textbf{#1}\\}
\renewcommand{\line}{\begin{tabularx}{\textwidth}{X>{\raggedleft}X}\hline\\\end{tabularx}\\[-0.5cm]}
\newcommand{\leftright}[2]{\begin{tabularx}{\textwidth}{X>{\raggedleft}X}#1%
& #2\\\end{tabularx}\\[-0.5cm]}
%\linespread{2} %-- Uncomment for Double Space
\begin{document}
\renewcommand{\headrulewidth}{0pt}
\fancyhf{}
\fancyhead[L]{
\leftright{\textbf{Logik}}{Hannes Strubel, Daniel Schmidt}
\line
\leftright{\textbf{Gruppe: Mo, 8:15 - 9:45}}{Serie 3}
}
\fancyfoot[C]{\thepage}

\section*{FOLGAL.2}
Aus dieser Menge folgt:\\
\begin{equation}
X_0
\end{equation}
und aus dieser Menge folgt nicht:\\
\begin{equation}
\neg X_3
\end{equation}

\section*{FOLGAL.3}
Falls dies der Fall ist kann man sagen, dass $\Phi$ nicht erfüllbar ist, denn nur aus falsch kann alles folgen.

\end{document}