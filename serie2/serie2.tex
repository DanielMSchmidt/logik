\documentclass[12pt, a4paper]{article}
\usepackage{url,graphicx,tabularx,array,geometry}
\usepackage[utf8]{inputenc}
\usepackage{paralist}
\usepackage{latexsym}
\usepackage{fancyhdr}
\usepackage{ textcomp }
\usepackage{ mathrsfs }
\usepackage{ dsfont }

\pagestyle{fancy}

\usepackage{amsmath}
\usepackage{amsfonts}
\usepackage{amssymb}


\setlength{\parskip}{1ex} %--skip lines between paragraphs
\setlength{\parindent}{0pt} %--don't indent paragraphs

%-- Commands for header
\newcommand{\bs}{\ensuremath{\backslash}}
\renewcommand{\title}[1]{\textbf{#1}\\}
\renewcommand{\line}{\begin{tabularx}{\textwidth}{X>{\raggedleft}X}\hline\\\end{tabularx}\\[-0.5cm]}
\newcommand{\leftright}[2]{\begin{tabularx}{\textwidth}{X>{\raggedleft}X}#1%
& #2\\\end{tabularx}\\[-0.5cm]}
%\linespread{2} %-- Uncomment for Double Space
\begin{document}
\renewcommand{\headrulewidth}{0pt}
\fancyhf{}
\fancyhead[L]{
\leftright{\textbf{Logik}}{Hannes Strubel, Daniel Schmidt}
\line
\leftright{\textbf{Gruppe: Mo, 8:15 - 9:45}}{Serie 2}
}
\fancyfoot[C]{\thepage}

\section*{AEAL.1}
\subsection*{Frage}
Welche der Formeln sind äquivalent?\\
$X_0 \vee \neg X_1, \neg(X_1 \vee \neg X_0), X_1 \rightarrow X_0, X_0 \uparrow X_1, \neg X_0 \leftrightarrow X_1$\\

\subsection*{Antwort}
Da zwei Formeln äquivalent sind wenn sie bei jeder Variablenbelegung den selben Wahrheitswert ergeben. Wie im folgenden zu sehen ist, ist die nirgends wer Fall. 
\begin{equation}
\textlbrackdbl \phi \textrbrackdbl_{\alpha} = \textlbrackdbl \vartheta \textrbrackdbl_{\alpha}
\end{equation}\\

\begin{tabular}{|l|l|l|l|}
\hline
$X_1$ & $X_0$ & $\neg X_1$ & $X_0 \vee \neg X_1$\\ \hline
0 & 0 & 1 & 1\\
0 & 1 & 1 & 1\\
1 & 0 & 0 & 0\\
1 & 1 & 0 & 1\\
\hline
\end{tabular}

\begin{tabular}{|l|l|l|l|l|}
\hline
$X_1$ & $X_0$ & $\neg X_0$ & $X_1 \vee \neg X_0$ & $\neg(X_1 \vee \neg X_0)$\\ \hline
0 & 0 & 1 & 1 & 0\\
0 & 1 & 0 & 0 & 1\\
1 & 0 & 1 & 1 & 0\\
1 & 1 & 0 & 1 & 0\\
\hline
\end{tabular}

\begin{tabular}{|l|l|l|}
\hline
$X_1$ & $X_0$ & $X_1 \rightarrow X_0$ \\ \hline
0 & 0 & 1\\
0 & 1 & 0\\
1 & 0 & 1\\
1 & 1 & 1\\
\hline
\end{tabular}

\begin{tabular}{|l|l|l|}
\hline
$X_1$ & $X_0$ & $X_0 \uparrow X_1$\\ \hline
0 & 0 & 1\\
0 & 1 & 1\\
1 & 0 & 1\\
1 & 1 & 0\\
\hline
\end{tabular}

\begin{tabular}{|l|l|l|l|}
\hline
$X_1$ & $X_0$ & $\neg X_0$ & $\neg X_0 \leftrightarrow X_1$\\ \hline
0 & 0 & 1 & 0\\
0 & 1 & 0 & 1\\
1 & 0 & 1 & 1\\
1 & 1 & 0 & 0\\
\hline
\end{tabular}
\end{document}