\documentclass[12pt, a4paper]{article}
\usepackage{url,graphicx,tabularx,array,geometry}
\usepackage[utf8]{inputenc}
\usepackage{paralist}
\usepackage{latexsym}
\usepackage{fancyhdr}
\usepackage{ textcomp }
\usepackage{ mathrsfs }
\usepackage{ dsfont }
\pagestyle{fancy}

\usepackage{amsmath}
\usepackage{amsfonts}
\usepackage{amssymb}


\setlength{\parskip}{1ex} %--skip lines between paragraphs
\setlength{\parindent}{0pt} %--don't indent paragraphs

%-- Commands for header
\newcommand{\bs}{\ensuremath{\backslash}}
\renewcommand{\title}[1]{\textbf{#1}\\}
\renewcommand{\line}{\begin{tabularx}{\textwidth}{X>{\raggedleft}X}\hline\\\end{tabularx}\\[-0.5cm]}
\newcommand{\leftright}[2]{\begin{tabularx}{\textwidth}{X>{\raggedleft}X}#1%
& #2\\\end{tabularx}\\[-0.5cm]}
%\linespread{2} %-- Uncomment for Double Space
\begin{document}
\renewcommand{\headrulewidth}{0pt}
\fancyhf{}
\fancyhead[L]{
\leftright{\textbf{Logik}}{Hannes Strubel, Daniel Schmidt}
\line
\leftright{\textbf{Gruppe: Mo, 8:15 - 9:45}}{Serie 2}
}
\fancyfoot[C]{\thepage}

\section*{AEAL.1}
\subsection*{Frage}
Welche der Formeln sind äquivalent?\\
$X_0 \vee \neg X_1, \neg(X_1 \vee \neg X_0), X_1 \rightarrow X_0, X_0 \uparrow X_1, \neg X_0 \leftrightarrow X_1$\\

\subsection*{Antwort}
Da zwei Formeln äquivalent sind wenn sie bei jeder Variablenbelegung den selben Wahrheitswert ergeben. Wie im folgenden zu sehen ist, ist die nirgends wer Fall.
\begin{equation}
\textlbrackdbl \phi \textrbrackdbl_{\alpha} = \textlbrackdbl \vartheta \textrbrackdbl_{\alpha}
\end{equation}\\

\begin{tabular}{|l|l|l|l|}
\hline
$X_1$ & $X_0$ & $\neg X_1$ & $X_0 \vee \neg X_1$\\ \hline
0 & 0 & 1 & 1\\
0 & 1 & 1 & 1\\
1 & 0 & 0 & 0\\
1 & 1 & 0 & 1\\
\hline
\end{tabular}

\begin{tabular}{|l|l|l|l|l|}
\hline
$X_1$ & $X_0$ & $\neg X_0$ & $X_1 \vee \neg X_0$ & $\neg(X_1 \vee \neg X_0)$\\ \hline
0 & 0 & 1 & 1 & 0\\
0 & 1 & 0 & 0 & 1\\
1 & 0 & 1 & 1 & 0\\
1 & 1 & 0 & 1 & 0\\
\hline
\end{tabular}

\begin{tabular}{|l|l|l|}
\hline
$X_1$ & $X_0$ & $X_1 \rightarrow X_0$ \\ \hline
0 & 0 & 1\\
0 & 1 & 0\\
1 & 0 & 1\\
1 & 1 & 1\\
\hline
\end{tabular}

\begin{tabular}{|l|l|l|}
\hline
$X_1$ & $X_0$ & $X_0 \uparrow X_1$\\ \hline
0 & 0 & 1\\
0 & 1 & 1\\
1 & 0 & 1\\
1 & 1 & 0\\
\hline
\end{tabular}

\begin{tabular}{|l|l|l|l|}
\hline
$X_1$ & $X_0$ & $\neg X_0$ & $\neg X_0 \leftrightarrow X_1$\\ \hline
0 & 0 & 1 & 0\\
0 & 1 & 0 & 1\\
1 & 0 & 1 & 1\\
1 & 1 & 0 & 0\\
\hline
\end{tabular}

\section*{AEUM.1}
\subsection*{Frage}
Zeigen Sie durch eine reine syntaktische Argumentation, dass\\
$\neg ((X_0 \rightarrow X_2) \rightarrow X_3) \wedge X_4 \equiv (X_4 \wedge \neg X_3) \wedge (X_0 \rightarrow X_2)$\\
gilt.
\subsection*{Antwort}
\begin{equation}
\begin{split}
\neg((X_0 \rightarrow X_2) \rightarrow X_3) \wedge X_4
\equiv&^1 \neg(\neg(X_0 \rightarrow X_2) \vee X_3) \wedge X_4\\
\equiv&^2 (\neg\neg(X_0 \rightarrow X_2) \wedge \neg X_3) \wedge X_4\\
\equiv&^3 ((X_0 \rightarrow X_2) \wedge \neg X_3) \wedge X_4\\
\equiv&^4 X_0 \rightarrow X_2 \wedge \neg X_3 \wedge X_4\\
\equiv&^5 (\neg X_3 \wedge X_4) \wedge (X_0 \rightarrow X_2)
\end{split}
\end{equation}

\begin{enumerate}
\item Elimination von Konditional
\item De Morgan
\item Doppelte Negation
\item Überflüssige Klammern weggelassen
\item Kommutativität
\end{enumerate}

\section*{AGAL.1}
\subsection*{Frage}
Beweisen Sie eines der obigen Gesetze. Führen Sie im Falle des letzten Gesetzes einen Beweis per vollständiger Induktion. Wir beweisen im folgenden die doppelte Negation.
\subsection*{Antwort}
\begin{tabular}{|l|l|l|}
\hline
$\vartheta$ & $\neg \vartheta$ & $\neg(\neg \vartheta)$\\ \hline
0 & 1 & 0\\
1 & 0 & 1\\
\hline
\end{tabular}

\section*{SUBAL.1}
\subsection*{Frage}
Wenden Sie die Substitution\\
$[X_{1} \longmapsto \neg (X_{2} \vee X_{3}), X_{3} \longmapsto X_{1}]$ auf die Formel
$ X_{0} \leftrightarrow (\neg X_{1} \uparrow (X_{2} \rightarrow \neg (X_{2} \nleftrightarrow X_{3})))$ an.\\
\subsection*{Antwort}
$\sigma = [X_{1} \longmapsto \neg (X_{2} \vee X_{3}), X_{3} \longmapsto X_{1}]$
\begin{equation}
\begin{split}
(X_{0} \leftrightarrow (\neg X_{1} \uparrow (X_{2} \rightarrow \neg (X_{2} \nleftrightarrow X_{3}))))\sigma 
\equiv&^1 (X_{0})\sigma \leftrightarrow (\neg X_{1} \uparrow (X_{2} \rightarrow \neg (X_{2} \nleftrightarrow X_{3})))\sigma\\ 
\equiv&^2 X_{0} \leftrightarrow ((\neg X_{1})\sigma \uparrow (X_{2} \rightarrow \neg (X_{2} \nleftrightarrow X_{3}))\sigma)\\
\equiv&^3 X_{0} \leftrightarrow (\neg(X_{1}\sigma) \uparrow (X_{2}\sigma \rightarrow (\neg (X_{2} \nleftrightarrow X_{3}))\sigma))\\
\equiv&^4 X_{0} \leftrightarrow (\neg \sigma(X_{1}) \uparrow (X_{2} \rightarrow \neg ((X_{2} \nleftrightarrow X_{3})\sigma)))\\
\equiv&^5 X_{0} \leftrightarrow (\neg (\neg (X_{2} \vee X_{3})) \uparrow (X_{2} \rightarrow \neg ((X_{2}\sigma \nleftrightarrow X_{3}\sigma))))\\
\equiv&^6 X_{0} \leftrightarrow ((X_{2} \vee X_{3}) \uparrow (X_{2} \rightarrow \neg ((X_{2}\sigma \nleftrightarrow X_{3}\sigma))))\\
\equiv&^7 X_{0} \leftrightarrow ((X_{2} \vee X_{3}) \uparrow (X_{2} \rightarrow \neg (X_{2} \nleftrightarrow \sigma(X_{3}))))\\
\equiv&^8 X_{0} \leftrightarrow ((X_{2} \vee X_{3}) \uparrow (X_{2} \rightarrow \neg (X_{2} \nleftrightarrow X_1)))\\
\end{split}
\end{equation}

\section*{MAE.1}
\subsection*{Frage}
Stellen Sie die Zweifärbbarkeitsformel zum Haus des Nikolaus und die Drei- sowie Vierfärbbarkeitsformeln zu einem Dreieck auf.
\subsection*{Antwort}
Vierfärbbarkeit:\\
\\
\\ %Hier nochmal reinzeichnen
\\
$ ((X_0 \nleftrightarrow X_2) \vee (X_1 \nleftrightarrow X_3)) \wedge ((X_2 \nleftrightarrow X_4) \vee (X_3 \nleftrightarrow X_5)) \wedge ((X_4 \nleftrightarrow X_0) \vee (X_5 \nleftrightarrow X_1))$\\
oder allgemein Formuliert\\
$\bigwedge_{i<3} ((X_{2i} \nleftrightarrow X_{2(i+1) mod 6}) \vee (X_{(2i)+1} \nleftrightarrow X_{2(i+1)+1 mod 6}))$\\
\\
Dreifärbbarkeit:\\
\\
\\ % Hier nochmal reinzeichnen
\\
$ ((X_0 \nleftrightarrow X_2) \vee (X_1 \nleftrightarrow X_3)) \wedge ((X_2 \nleftrightarrow X_4) \vee (X_3 \nleftrightarrow X_5)) \wedge ((X_4 \nleftrightarrow X_0) \vee (X_5 \nleftrightarrow X_1)) \wedge (X_0 \uparrow X_1) \wedge (X_2 \uparrow X_3) \wedge (X_4 \uparrow X_5)$\\
oder allgemein Formuliert\\
$\bigwedge_{i<3} /((X_{2i} \nleftrightarrow X_{2(i+1) mod 6}) \vee (X_{(2i)+1} \nleftrightarrow X_{2(i+1)+1 mod 6})) \wedge (X_{2i} \uparrow X_{2i+1}) )$\\
\\
Haus des Nikolaus:\\
\\
\\ % Hier nochmal reinzeichnen
\\
$(X_0 \nleftrightarrow X_1) \vee (X_1 \nleftrightarrow X_2) \vee (X_2 \nleftrightarrow X_3) \vee (X_4 \nleftrightarrow X_0) \vee (X_0 \nleftrightarrow X_2) \vee (X_1 \nleftrightarrow X_4)$\\
oder allgemeiner Formuliert\\
$(\bigwedge_{i<5} (X_i \nleftrightarrow X_{i+1})) \wedge (X_0 \nleftrightarrow X_2) \wedge(X_1 \nleftrightarrow X_4)$
\end{document}